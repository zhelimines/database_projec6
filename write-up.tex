\documentclass{article}
\title{Write Up}
\author{Brandon Parrish}
\date{May 2, 2016}

\usepackage{hyperref}

\begin{document}
	\maketitle

	Our data is the (approximately) 1,000,000 most frequent 5-grams from the Corpus of Contemporary American English (COCA). The source for this from \url{http://www.ngrams.info}. To download the data we had to give and email and name and agree not "distribute this list to others, and to not develop any other frequency lists that are based on this data, which will be sold commercially." 
\vspace{5mm}
	The data we use contains eleven attributes all together, but we only use six attriutes for our word correction algorithm. Our data consist of a frequency attribute and a 5-gram that is split into 5 separate string attributes per row. The frequency represents the relative number of times the 5-gram is used in comparison to the rest of the 5-grams. 

\newpage 
\section{Additional Links}

http://norvig.com/ngrams/ (list of dataset files) \\

http://norvig.com/spell-correct.html (how to implement a basic spellchecker) \\

https://datahub.io/dataset/beautiful-data-natural-language-corpus-and-code (download for files) \\

ota.ox.ac.uk/headers/0643.xml (spelling error corpus) \\

aspell.net/test/ (possibly better spelling error corpus) \\

www.ngrams.info/download\_coca.asp (corpus of contemporary english, 2-, 3-, 4-, and 5-grams) \\

www.postgresql.org/docs/9.5/interactive/textsearch.html (full text search if we want to use stemming or something) \\

\end{document}
